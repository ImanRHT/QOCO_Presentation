\section{Conclusion}
%\subsection{Markov Decision Process}

\begin{frame}
	\frametitle{Conclusion}
	


	\begin{itemize}[]
		
		\item We formulated an optimization problem that aims to maximize the QoE of each MD individually.
		\begin{itemize}
			\item The QoE reflects the energy consumption and task completion delay.
		\end{itemize}
		
		
		\item Empowering MDs to make offloading decisions.
		\begin{itemize}
			\item (without relying on knowledge about task models or other MDs' offloading decisions)
		\end{itemize}
		 
		
		\item  Adapts to the uncertain dynamics of load levels at ENs.
		\begin{itemize}
			\item Effectively manages the ever-changing system environment.
		\end{itemize} 
		
	\end{itemize}

\vfill

	\textbf{Future Work:}

	\begin{itemize}[]
	
	\item Extending the task model by considering interdependencies among tasks. 
	\begin{itemize}
		\item This can be achieved by incorporating a \textbf{task call graph representation} to develop dependency among task partitions.
	\end{itemize}
	
	\item Enabling MDs to take advantage of federated learning techniques in the training process. 
	\begin{itemize}
		\item This allows MDs to collectively contribute to improving the offloading model and enable continuous learning when new MDs join the network.
	\end{itemize}

	
\end{itemize}
	
\end{frame}


